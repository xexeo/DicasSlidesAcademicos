\documentclass[aspectratio=169]{beamer}

 % Altere o tema se preferir
\usepackage[brazil]{babel} % Suporte para português brasileiro

\usepackage[%
numbering=fraction,%
block=fill,%
subsectionpage=progressbar]{pescPres}



\usepackage{lipsum} % Para gerar texto de exemplo (Lorem Ipsum)
\usepackage{graphicx} % Para adicionar imagens
\usepackage{multicol} % Para layouts em várias colunas

\title{Exemplo de Apresentação Beamer}
\subtitle{Configurações de Slides}
\author{Seu Nome}
\advisor{Colocar o Nome do Orientador} % opcional
\date{\today}

\begin{document}



% Slide 1: Slide de Título
\begin{frame}
    \titlepage
\end{frame}

\section{Primeira seção}

% Slide 2: Lista com marcadores
\begin{frame}{Lista com Marcadores}
    \begin{itemize}
        \item Primeiro ponto
        \item Segundo ponto
        \item Terceiro ponto
           \end{itemize}
\end{frame}

% Slide 3: Lista numerada
\begin{frame}{Lista Numerada}
    \begin{enumerate}
        \item Primeiro item
        \item Segundo item
        \item Terceiro item

    \end{enumerate}
\end{frame}

% Slide 4: Duas colunas com marcadores
\begin{frame}{Duas Colunas com Marcadores}
    \begin{columns}
        \column{0.5\textwidth}
            \begin{itemize}
                \item Coluna 1, Item 1
                \item Coluna 1, Item 2

            \end{itemize}
        \column{0.5\textwidth}
            \begin{itemize}
                \item Coluna 2, Item 1
                \item Coluna 2, Item 2

            \end{itemize}
    \end{columns}
\end{frame}

% Slide 5: Slide com imagem
\begin{frame}{Slide com Imagem}
    \begin{center}
        \includegraphics[width=0.5\textwidth]{example-image} % Substitua pela sua própria imagem
    \end{center}
\end{frame}

% Slide 6: Texto e imagem lado a lado
\begin{frame}{Texto e Imagem Lado a Lado}
    \begin{columns}
        \column{0.5\textwidth}
            Esse é um texto sobre a imagem do lado
        \column{0.5\textwidth}
            \includegraphics[width=\textwidth]{example-image} % Substitua pela sua própria imagem
    \end{columns}
\end{frame}

\section{Segunda Seção}

% Slide 7: Slide com tabela
\begin{frame}{Exemplo de Tabela}
    \begin{table}[htbp]
        \centering
        \begin{tabular}{|c|c|c|}
        \hline
        Cabeçalho 1 & Cabeçalho 2 & Cabeçalho 3 \\
        \hline
        Célula 1  & Célula 2  & Célula 3  \\
        Célula 4  & Célula 5  & Célula 6  \\
        \hline
        \end{tabular}
        \caption{Esta é uma tabela de exemplo.}
    \end{table}
\end{frame}

% Slide 8: Texto em várias colunas
\begin{frame}{Texto em Várias Colunas}
    \begin{multicols}{2}
        \lipsum[2]
    \end{multicols}
\end{frame}

% Slide 9: Bloco de texto
\begin{frame}{Bloco de Texto}
    \begin{block}{Informação Importante}
        Esse é um texto dentro de um bloco de texto que pode ser configurado de várias maneiras.
    \end{block}
\end{frame}

% Slide 10: Slide de citação
\begin{frame}{Slide de Citação}
    \begin{quote}
        \LaTeX é uma ferramenta muito boa e muito ruim.
    \end{quote}
\end{frame}

% Slide 11: Caixa de destaque
\begin{frame}{Exemplo de Caixa de Destaque}
    \begin{alertblock}{Ponto Chave}
       Um ponto chave sobre a apresentação
    \end{alertblock}
\end{frame}

% Slide 12: Slide de conclusão
\begin{frame}{Conclusão}
    \begin{itemize}
        \item \textbf{Resumo:} Esta é a conclusão da apresentação.
        \item Concluído agora.
    \end{itemize}
\end{frame}

\begin{frame}[plain]
 sem cabeçalho
\end{frame}

\begin{frame}[standout]
	Invertido!
\end{frame}

\begin{frame}[allowframebreaks]
    \frametitle{Slide longo}
    \begin{itemize}
        \item Primeiro ponto
        \item Segundo ponto
        \item Terceiro ponto
        \item 1 ponto
        \item 2 ponto
        \item 3 ponto
        \item 4 ponto
        \item 5 ponto
        \item 6 ponto
        \item 7 ponto
        \item 8 ponto
        \item 9 ponto
        \item 10 ponto
        \item 11 ponto
        \item 12 ponto
        \item 13 ponto

    \end{itemize}
\end{frame}




\end{document}
