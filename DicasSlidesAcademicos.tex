\documentclass[12pt,a4paper]{article}
\usepackage[utf8]{inputenc}
\usepackage[T1]{fontenc}
\usepackage[brazilian]{babel}
\usepackage{amsmath}
\usepackage{amsfonts}
\usepackage{amssymb}
\usepackage{graphicx}
\usepackage{outlines}
\usepackage{indentfirst}
\usepackage{datetime}
\setlength{\parskip}{.5em}
\usepackage{enumitem}
\setlist{nosep}
\usepackage[section]{placeins}
\def\tam{0.6}
\usepackage{draftwatermark}
\SetWatermarkText{DRAFT}
\SetWatermarkScale{1}
\usepackage[all]{nowidow}
\nowidow
\noclub
\usepackage[scale=1.5]{ccicons}
\usepackage{authblk}
\usepackage{subfig}
\usepackage{listings}
\usepackage{hyperref}



\title{Dicas para Provas de Aula}
\author{Geraldo Xexéo}
\affil{\url{xexeo@ufrj.br} \\
    \url{http://xexeo.net}}
\date{\ccbyncsa\  - \today}


\author{Geraldo Xexéo}
\title{Dicas de Como Fazer Slides Acadêmicos}

\begin{document}



\maketitle

\begin{abstract}
    Este artigo fornece dicas de como fazer slides de aula, para apresentações em congressos e para defesas de projeto final, dissertações, teses e exames de qualificação.
\end{abstract}

\tableofcontents


\section{Introdução}

Este documento é um guia com orientações para o uso de slides em apresentações a serem feitas pelos laboratórios LUDES e LINE, no Programa de Engenharia de Sistemas e Computação da COPPE/UFRJ. Devem ser úteis a outros grupos.

Ele é construído como um guia básico, e razoavelmente conservador.
Além disso, ele não é um guia de \textit{design}.
Apresentações que seguem essas regras podem ser muito mais bonitas.

Este material está disponível no GitHub, junto com material de apoio, como slides Power Point que eu uso e distribuo: \url{https://github.com/xexeo/DicasSlidesAcademicos}.

Outra questão importante neste documento: ele é criado por um usuário do \textit{Power Point} que tem pouca experiência com o \texttt{beamer}, o estilo de apresentações do \LaTeX.
Muitos softwares de apresentação são altamente compatíveis com o \textit{Power Point}, seguindo a mesma filosofia de trabalho, baseado no WYSIWYG.
Outros softwares, como o Prezi\footnote{\url{https://prezi.com/}} podem exigir outra forma de pensar, porém muitas dicas dadas aqui continuam válidas. A Seção \ref{sec:ferramentas} discute um pouco os principais softwares e características importantes do seu uso.

Usarei o termo geral \textit{apresentação} para uma aula, defesa de tese ou apresentação de artigo em congresso.

Os slides são impressos com frames gerados no \LaTeX\ para delimitar o seu tamanho, os frames não fazem parte dos mesmos, podendo ser gerados de outra forma por outros aplicativos.

Mais uma questão para o leitor levar em consideração: todos os slides apresentados são reais, e quebram algumas vezes minhas sugestões.
Por exemplo, os slides feitos com \texttt{beamer} usam números pequenos, o que é uma característica do tema Lubeck.
Já os slides de \textit{Power Point} muitas vezes não possuem o número total de slides, porque isso dá mais trabalho do que deveria nesse software.

\subsection{Filosofia dos Slides}

Os ``slides'' são uma ferramenta usada em apresentações há muitos anos, sendo substitutos digitais de outras formas de apresentação, como quadro negro, posters, fotografias, projeções com slides (do tipo filme), transparências, etc.
Muitos autores famosos questionam o uso de softwares como o PowerPoint para a criação de slides, ou mesmo para a apresentação de projeto e aulas, porém não há como negar que eles são práticos, facilmente criados e usados universalmente.

Este guia segue uma filosofia: o trabalho de criar os slides não pode ser grande.
Além disso, eles devem ajudar tanto o professor a passar sua mensagem, mas também ao aluno a acompanhar a aula, entender o material, e poder usá-lo mais tarde.

Para isso, os slides não podem ser nem entendiantes, nem levar o aluno a se dispersar, ou se perder tamanha a quantidade de informações.
Isso se aplica tanto a um slide isolado, quanto a uma sequência de slides.

Além disso, os slides são para guiar a platéia e não devem conter grande quantidade de texto.
O apresentador não lê o slide ao apresentar, mas apenas se guia.

Além disso, devem ser criados para atender não só um assistência normal, mas também os que tem alguma dificuldade, como visualização de cores, dificuldade de leitura a distância e mesmo dificuldades auditivas (nesse caso, o slide deverá conter informação suficientes para ajudar que assiste a entender o assunto).

Nossa filosofia parte da ideia de 4 tipos de slides padrão: o de título, o slide de texto, o slide de imagem e o slide de texto e imagem.
A partir desses tipos, várias variações podem ser criadas.

\begin{figure}[hbt]
    \centering
    \subfloat{
        \includegraphics[width=0.4\linewidth,frame]{Slides/tiposbasicos/Slide1}}
    \subfloat{
        \includegraphics[width=0.4\linewidth,frame]{Slides/tiposbasicos/Slide2}}\\
    \subfloat{
    \includegraphics[width=0.4\linewidth,frame]{Slides/tiposbasicos/Slide3}}
    \subfloat{
    \includegraphics[width=0.4\linewidth,frame]{Slides/tiposbasicos/Slide4}}
    \caption{Tipos básicos de slides.}
    \label{fig:tiposbasicos}
\end{figure}

Outra premissa desse texto, que dificilmente pode ser discutida, é que a apresentação é feita, pelo professor ou apresentador, na forma de uma narrativa linear.
Mesmo que use truques de apresentar algo antes, ou deixar algo para mais tarde, de qualquer forma, a narrativa será feita, durante a apresentação, passo a passo.
Voltar a trás, dar ``hiperpulos'' com o texto só causa confusão a plateia.

\subsection{Outras ferramentas}

Existem outras ferramentas que propõe o que seriam novas formas de fazer apresentações, organizando a apresentação de forma não linear, como o Prezi.
Como vimos, apresentações são obrigatoriamente narrativas lineares, e devem ser construídas dessa forma, logo, não faz sentido construi-las de outra maneira.
A abordagem hipertextual, nesse caso, confunde a audiência.
\section{Conteúdo dos slides}

Um slide típico, no caso geral, tem duas funções básicas, uma ligada ao momento em que ele é usado e outra ligada ao seu uso futuro.

No momento em que é usado, o slide deve servir de referência para o assunto e também para indicar o momento da apresentação, posicionando a audiência. Assim, o número do slide serve para mostrar quanto o apresentador está avançado, o título indica o assunto sendo tratado e os itens textuais e imagens mostram o detalhamento do assunto.



Então, no primeiro uso, é importante trabalhar o que está sendo apresentado naquele momento, que indica um intervalo de tempo onde o slide será apresentado, dentro de toda apresentação. Dependendo do tipo de apresentação, isto pode ser feito de várias formas. Por exemplo, uma apresentação mais motivacional pode colocar apenas perguntas nos slides, e o apresentador pode respondê-las enquanto fala. Já uma apresentação formal de tese exige que o conteúdo esteja lá para referência da banca.

Já para o segundo uso, como uma leitura posterior, temos que trabalhar a qualidade da informação sendo passada. Algumas apresentações não são feitas com essa intenção, como é o caso de algumas apresentações motivacionais, mas no mundo acadêmico é uma norma geral que os slides possam ser usados como uma referência futura ao tópico, pelo menos de forma inicial.

O conteúdo típico de um slide é um lista de itens  que representa o que vai ser falado naquele momento. A menos de definições formais e informações específicas, como tabelas de dados, o conteúdo não deve ser lido de forma explícita, mas sim ser uma forma reduzida do que é falado.  Essa lista pode conter exemplos, definições, motivação, dependendo da necessidade do slide.

Por exemplo, o parágrafo anterior poderia ser dito, de forma semelhante, em um slide com o título ``Conteúdo Básico de Um Slide'':
\begin{itemize}
    \item lista de itens tratados;
    \item definições formais;
    \item tabelas e gráficos, e
    \item resumo do conteúdo da fala.
\end{itemize}

O slide da Figura \ref{fig:coppe} é um slide típico criado no \textit{Power Point}. Ele contém um título, que no caso se refere a uma categoria introduzida no slide anterior, e quatro subcategorias, que são explicadas em letras menores.

\begin{figure}[htb]
    \centering
    \includegraphics[width=\tam\linewidth,frame]{imagens/slideexemplotepj.png}
    \caption{Um slide com logo, identificação do autor (incluindo e-mail) ,nome do curso,   número do slide e número total de slides}
    \label{fig:coppe}
\end{figure}

Chamo a atenção que se a apresentação, seja ela uma aula, uma defesa de tese ou de outro tipo, só contiver slides como o da Figura \ref{fig:coppe},  será monótona. É importante variar a forma de apresentar o conteúdo e o estilo do slide, assunto que será tratado mais a frente nesse artigo.

O slide da Figura \ref{fig:dados} é um exemplo típico de um slide monótono, mas algumas vezes usado em uma aula. O professor tem que apresentar uma lista de propriedades, no caso dos dados derivados. Talvez um slide para cada item fosse mais adequado, mas isso poderia levar a uma apresentação muito longa. A questão importante aqui é balancear o conteúdo, o slide, e o tempo em que ele aparece. Esse slide pode ser falado em um segundo, se serve apenas de introdução aos vários assuntos, em poucos minutos, se alguns assuntos vão ser tratados mais detalhadamente e outros não, ou pode tomar uma hora, o que seria uma péssima opção para o ritmo da aula.

Na verdade, ao encontrar este slide eu me perguntei se ele é adequado ou deveria ser ``refatorado''. Lembro que são 2 slides, com um total de 19 itens a serem discutidos apenas sobre as características dos dados derivados. Antes, outras 18 características foram usadas, em 2 slides, para descrever dados primitivos. Como tratar isso de forma melhor? Como evitar 4 slides monótonos que levam a uma aula provavelmente longa demais? Essas são perguntas típicas que temos que nos fazer ao criar qualquer apresentação, para evitar ``perder a plateia''. Conteúdo e forma são aliados para resolver problemas desse tipo.

\begin{figure}[tbh]
    \centering
    \includegraphics[width=\tam\linewidth,frame]{imagens/dados}
    \caption{Um slide com uma lista de itens. Como a lista é longa, foi dividido em dois.}
    \label{fig:dados}
\end{figure}


A Figura \ref{fig:tres} mostra uma sequência de slides sobre um mesmo tema: riscos de software que se tornaram realidade. Ao invés de colocar vários slides com listas de riscos, optei por usar primeiro uma explicação detalhada passo a passo do acidente do Arianne V, depois uma coleção de notícias recortadas do jornal, que é narrada mais rapidamente, e só no final um slide simples tradicional. Noto que, tanto na notícia do Arianne V quanto nos desastres financeiros, os itens aparecem um a um para a assistência, o que facilita a criação de uma narrativa. Chamo isso de dinâmica do slide, feita na forma de uma animação, e discuto na Seção \ref{sec:din}. O acidente do Arianne V também é mostrado com uma imagem que chama a atenção, servindo para o professor contar a história e chamar atenção dos alunos para a aula.

\begin{figure}
    \centering
    \subfloat{
    \includegraphics[width=0.3\linewidth,frame]{imagens/arianne.png}
}
   \subfloat{    \includegraphics[width=0.3\linewidth,frame]{imagens/hell.png}
    }
\subfloat{
    \includegraphics[width=0.3\linewidth,frame]{imagens/problemas.png}
}
\caption{Três slides tratando do mesmo assunto, riscos de software, onde conteúdo e estilo são diferentes. Mesmo assim, eles podem ser percebidos como pertencendo a mesma aula, devido as características comuns do estilo.}
\label{fig:tres}
\end{figure}


É importante que tudo que seja colocado no slide, seja também referenciado de alguma forma, seja pela bibliografia, seja por citações específicas em algum local do slide.
Forneça todas as referências, e \textbf{indique a propriedade intelectual de tudo}. Prefira imagens de domínio público ou com licenças amplas, como \textit{Creative Commons}.


\subsection{Slide de contato}

Como adicional, toda apresentação deve possuir um slide final que indica um contato. Hoje, todas as minhas aulas terminam com o slide da Figura \ref{fig:fim}. Isso pode ser usado sempre para falar algo como ``quem quiser me contatar para tirar dúvidas...''.

\begin{figure}[h]
    \centering
    \includegraphics[width=\tam\linewidth,frame]{imagens/fim.png}
    \caption{Um slide de contato.}
    \label{fig:fim}
\end{figure}

\section{O estilo dos slides}

Os slides devem apresentar uma identidade conjunta. Para isso devem ser usados estilos apropriados, que estão disponíveis nas ferramentas de criação, ou se criar um estilo novo.

Esse estilo deve possuir vários tipos de slides. A aula deve usar mais de um desses tipos, tanto para cumprir papéis posicionais, como o título, quanto para não ficar monótona. Os tipos principais são:
\begin{itemize}
    \item título;
    \item o título de seção;
    \item o slide de uma coluna, o mais comum;
    \item o slide de duas colunas;
    \item o slide de duas colunas com títulos, e
    \item o slide só com título, usado para figuras.
\end{itemize}


A Figura \ref{fig:tiposbasicosdopp}, copiada do \textit{Power Point} mostra esses seis tipos principais e mais alguns disponíveis para uma apresentação em branco, como o slide branco, e dois modelos com legenda.

\begin{figure}[htb]
    \centering
    \includegraphics[width=0.5\linewidth]{imagens/tiposbasicosdopp}
    \caption{Slides básicos disponíveis no Power Point}
    \label{fig:tiposbasicosdopp}
\end{figure}

Cada tipo de uso, como apresentação, aula, defesas ou exames, tem um tipo de slide mais adequado, de acordo com a necessidade de chamar atenção, e o grau de formalidade.

Em qualquer tipo de uso, porém, existem alguns objetos que devem aparecer nos slides, como a numeração e a identificação do autor e da instituição.

Os slides \textbf{devem ser numerados} e conter em cada slide o número total de slides, possivelmente no formato ``slide/total'', como em ``4/40''. Os números não podem ser pequenos, e eu favoreço números grandes, para que fique bem claro e possam, mesmo a distância, serem usados como referência. Esse número fica normalmente no pé (\textit{footer}) do cabeçalho.

Também é importante ter a identificação do autor. Normalmente ela inclui um e-mail ou um site.

Além disso, é interessante que, para a maioria dos usos, o estilo do slide esteja diretamente associado a uma instituição. Isso pode ser feito por meio da colocação do logo da instituição em uma posição clara.

No Power Point existem, por \textit{default}, três espaços no pé do slide (\textit{footer}). Um é reservado para o número. Os outros dois são possívelmente livres, sendo que um é sempre devemos usar identificar o autor. O terceiro espaço pode ser usado para o título da apresentação, o título do curso, o título do evento ou outra informação similar que se quer ressaltar.

A Figura \ref{fig:coppe} mostra um slide com todos esses elementos: o logo do PESC, o nome e e-mail do autor, o nome do curso, o número do slide em uma fonte grande e o número total de slides em uma fonte menor.

\begin{figure}[htb]
    \centering
    \includegraphics[width=\tam\linewidth]{imagens/slideexemplotepj.png}
    \caption{Um slide com logo, identificação do autor (incluindo e-mail) ,nome do curso,   número do slide e número total de slides}
    \label{fig:coppe}
\end{figure}

Use fontes ``limpas'', não rebuscadas, e \textbf{sem-serifa}\footnote{Serifas são as pontinhas que existem em algumas fontes. Elas estão bem visíveis no S da palavra ``slides'' desta seção.}, como Arial ou Calibri, e \textbf{corpos grandes}, 32 pts, por exemplo. Os slides das Figuras \ref{fig:coppe} e \ref{fig:teximag} seguem essa regra. Já o slide da figura \ref{fig:formulas} usa um tamanho menor para o corpo das fórmulas. Lembre que a banca, ao invés dos alunos, é mais velha e pode ter dificuldades de visão.


\subsubsection{Usando os logos corretos}

É importantíssimo usar os logos corretos das instituições. Para isso procure os logos originais e os manuais de marca.

A lista de logos que eu uso é:
\begin{itemize}
    \item UFRJ: \url{https://ufrj.br/comunicacao/manuais-e-modelos/marca-da-ufrj/}
    \item COPPE: \url{https://www.coppe.ufrj.br/pt-br/a-coppe/uso-da-marca}
    \item PESC:  \url{https://www.cos.ufrj.br/index.php/pt-BR/logo-pesc}
    \item IM: não fornece o logotipo na página, porém é possível copiar. A história da marca principal está em \url{https://sites.google.com/matematica.ufrj.br/mapcabral/outros/hist%C3%B3ria-do-logotipo-do-im}
    \item DCC: não fornece o logotipo na págica, mas, de qualquer maneira, será transformado no IC, com novo logotipo
    \item POLI: \url{http://www.poli.ufrj.br/marcadapolitecnica.php}
    \item LUDES: \url{https://github.com/LUDES-PESC/Generico/tree/master/Logo%20Novo%20Vers%C3%B5es}
    \item LINE: \url{https://github.com/LINE-PESC/Generico/tree/master/Logomarca%20LINE}
\end{itemize}

No GitHub deste documento estão disponíveis algumas sugestões de slides.


\subsection{Nomeando os slides}

Todo slide deve ter um \textbf{título único}. Esse espaço já vem reservado nos estilos de \textit{Power Point}.

Algumas pessoas, erroneamente, usam um título de seção que se repete nessa posição e colocam o que seria o título do slide como uma caixa-de-texto, ou como primeiro item da lista de itens do slide. Essa prática faz com que a audiência se perca em relação a onde o apresentador está. O nome e o número do slides servem não só para identificá-los, mas também como posicionamento na sequência.

É possível criar um slide com o nome da seção, mas ele deve ser menor que o nome do slide. Usando o \texttt{beamer}, o formato de slides do \LaTeX, é possível colocar no topo do slide uma mini-agenda, onde o nome da seção tem uma ênfase. A Figura \ref{fig:meio} mostra um slide que tem todas as seções identificadas em seu cabeçalho, sendo que a seção atual está com ênfase.

\begin{figure}[hb]
    \centering
    \includegraphics[width=\tam\linewidth]{imagens/agendadomeio.png}
    \caption{Um slide mostrando a parte que será falada da agenda, com as outras partes acinzentadas. Criada usando o \LaTeX\  e o \texttt{beamer}.}
    \label{fig:meio}
\end{figure}



\subsection{Variando e inventando}


É importante variar o estilo do slide. Isso é bem fácil no \textit{Power Point}, porém é mais difícil no \texttt{beamer}, por exemplo. A Figura \ref{fig:man} mostra um slide bem diferente do que os apresentados normalmente, mas ainda em um formato ``retangular''. Use os formatos para tirar a monotonia da aula. Use também animações nos slides, mas cuidado com as transições entre os slides, que devem ser usadas muito parcimoniosamente, porque quebram a atenção.

\begin{figure}[htb]
    \centering
    \includegraphics[width=\tam\linewidth]{imagens/manmonth.png}
    \caption{Um slide com um formato diferente}
    \label{fig:man}
\end{figure}


Os slides não devem ser exagerados, nem em texto, nem em decoração, porém um ou outro slide pode ser mais divertido, ou mais pesado em texto.

Em um slide com fórmulas, como o da Figura \ref{fig:formulas}, elas devem aparecer uma a uma se estiverem sendo calculadas. Se for apenas um comentário sobre a complexidade das fórmulas, que você deseje passar por cima em busca de uma explicação mais fácil, elas podem aparecer todas de uma vez.

\begin{figure}[hbt]
    \centering
    \includegraphics[width=\tam\linewidth]{imagens/desenhoeformulas.png}
    \caption{Desenho e fórmulas em um slide, que possui o logo do laboratório ligado ao curso e um logo que foi criado para identificar o curso em 3 lugares: Moodle, Whatsapp e GitHub.}
    \label{fig:formulas}
\end{figure}





Slides ``divertidos'', como os que estão resumidos\footnote{Esses slides foram encontrados em     \url{https://unblast.com/funtastic-free-powerpoint-presentation-template-ppt/}
} na Figura \ref{fig:fun} vão criar uma carga cognitiva muito grande em uma apresentação e podem incomodar membros da banca. Já vi isso acontecer. Mas isso não quer dizer que não possam ser usados em um ou outro slide, como marcos de início de seção ou outra alternativa de menor impacto que usá-los em toda aula.

\begin{figure}[hbt]
    \centering
    \includegraphics[width=\tam\linewidth]{imagens/funslide.jpg}
    \caption{Exemplos de slides divertidos.
        (Fonte: unblast.com) }
    \label{fig:fun}
\end{figure}

\section{Uso de Seções}

A apresentação deve ser dividida em seções. As seções são apresentadas na agenda, e, de preferência, devem variar entre 5 e 7, não sendo menos que 3 e não ultrapassando 9.

Na maioria das apresentações, duas seções são essenciais: introdução e conclusão. Algumas partes da apresentação, mesmo que sendo obrigatória sua presença, não precisam ter seções indicadas, como a Bibliografia e, quando existir, a apresentação do autor, autores ou grupo de pesquisa. A agenda não é um sumário do documento apresentação, mas sim uma divisão da apresentação em etapas.

Cada seção deve possuir uma slide inicial, na forma de título ou de uma versão da agenda onde a versão atual está indicada de alguma forma. Um título de seção pode ter vários formatos, a Figura \ref{fig:sectitles} mostra um simples e três variações que experimentei em aulas diferentes. Em especial, a agenda do slide da Figura \ref{fig:stfancy} usa uma \textit{SmartArt} do \textit{Power Point} e permite usar imagens e funciona bem até 5 seções. Tanto esse formato, quanto do da Figura \ref{fig:st3} mostram toda a agenda de alguma forma.

\begin{figure}[htb]
    \centering
    \subfloat[]{
        \includegraphics[width=0.4\linewidth,frame]{imagens/sec1}
    }
    \subfloat[]{
        \includegraphics[width=0.4\linewidth,frame]{imagens/valor}
    }\\
    \subfloat[]{
        \includegraphics[width=0.4\linewidth,frame]{imagens/casosdeuso}\label{fig:st3}
    }
    \subfloat[]{
        \includegraphics[width=0.4\linewidth,frame]{imagens/sec2}
        \label{fig:stfancy}}
        \caption{Várias formas de fazer um título de seção}.
        \label{fig:sectitles}
    \end{figure}


 A Figura \ref{fig:meio} mostra um slide com uma versão da agenda levemente modificada por meio do uso de sombra nos itens já tratados, que é criada automaticamente no \texttt{beamer} com os comandos que aparecem na Listagem \ref{lst:autosec}.

\begin{lstlisting}[language=TeX,caption={Comando para títulos de seção automáticos no \texttt{beamer} com o tema Luebeck.},label={lst:autosec}]
    \AtBeginSection[]
    {\begin{frame}
            \frametitle{Onde Estamos?}
            \tableofcontents[currentsection,hideallsubsections ]
    \end{frame}}
\end{lstlisting}

\begin{figure}[hbt]
    \centering
    \includegraphics[width=\tam\linewidth,frame]{imagens/agendadomeio.png}
    \caption{Um slide mostrando a parte que será falada da agenda, com as outras partes acinzentadas. Criada usando o \LaTeX\  e o \texttt{beamer}.}
    \label{fig:meio}
\end{figure}





\section{Dinâmica dos slides}
\label{sec:din}

Os slides podem ser tornados mais dinâmicos por meio de animações e transições. Uma animação ocorre dentro do slide, com os itens e imagens, já uma transição ocorre entre slides.

A princípio as transições devem ser usadas com muito cuidado, pois elas tem um impacto forte na atenção da audiência. Elas devem ser usadas apenas quando fazem sentido em relação ao fluxo ou ao conteúdo da apresentação. Considero que transições são praticamente proibidas em apresentações mais formais como defesas de tese.

Já as animações são um efeito muito útil, mas muitas vezes abusadas pelos apresentadores. Por exemplo, em uma lista de itens, como a da Figura \ref{fig:coppe}, alguns apresentadores poderiam escolher apresentar item a item e até mesmo tirar a cor dos itens já apresentados. Isto está errado, pois a platéia perde a visão global do assunto e a relação entre os itens a serem falados. Não devemos tratar a plateia como seres incapazes de separar um item do outro.

Os efeitos  básicos de ``aparecer'' ou ``desaparecer'', de várias formas, devem ser usados para a construção de raciocínio. Um exemplo típico é o uso do efeito de aparecer quando uma fórmula está sendo derivada.

O efeito de ``aparecer'' é muito útil na solução de exercícios. Ele permite que os itens sejam mostrados passo a passo na apresentação, mas estejam todos no slide final, que pode ser impresso ou colocado em PDF. O slide da Figura \ref{fig:canvas} usa o efeito de aparecer para posicionar os \textit{post-its} em um canvas, um a um. Isso é feito durante a resolução de um exemplo. Todos os \textit{post-its} se mantém na imagem, como acontece no uso do canvas no mundo real. Nesse caso, porém, nem todos são totalmente legíveis, o que prejudica o uso final dos slides.


\begin{figure}[htb]
    \centering
    \includegraphics[width=\tam\linewidth,frame]{imagens/canvas}
    \caption{Esse slide usa o efeito de ``aparecer'' para mostrar os \textit{post-its} um a um.}
    \label{fig:canvas}
\end{figure}


O efeito de aparecer também pode ser simulado usando vários slides. Isso é uma estratégia essencial se os slides forem usados no formato PDF.


Aliás, o efeito de desaparecer deve ser usado com mais cuidado do que o de aparecer, já que a informação de que algo foi tirado do slide pode ser importante para entender o contexto global quando o slide é visto após a apresentação, ou quando alguém chega a apresentação no meio do slide. Em alguns casos é melhor usar o efeito de retirar a cor ou colocar alguma marca, como um X, sobre o que iria ser retirado do slide.

A Figura \ref{fig:passos} mostra dois passos em sequência de uma animação apresentada slide a slide enquanto o professor descreve como problemas de comunicação podem fazer um sistema não ser aceito pelos usuários. Todos os balões da primeira imagem vão aparecendo ao longo da narrativa, enquanto o professor fala e clica. Eles são mantidos na imagem para mostrar que as ideias e conversas continuam ao longo do projeto, além de facilitar o entendimento geral, e letras são usadas para quem olhar para dar alguma noção de ordem. Porém, quando o sistema é entregue, e as conversas ``param'', uma mudança ocorre e todos os balões são apagados. Este me parece um uso adequado do mecanismo de desaparecimento.

\begin{figure}[hbt]
    \centering
    \subfloat{
    \includegraphics[width=0.4\linewidth,frame]{imagens/passo1}}
    \subfloat{
    \includegraphics[width=0.4\linewidth,frame]{imagens/passo2}}
    \caption{Dois passos de uma animação que mostra como um software pode dar errado.}
    \label{fig:passos}
\end{figure}

Já a Figura \ref{fig:beamer} mostra como o beamer gera uma animação, a partir do código exemplo da Listagem \ref{list:beamer}. Nesse caso, foi usada a opção de mostrar e apagar os itens, já que se apresentava uma interface de programa e seria contraproducente manter muita informação na tela.

\begin{figure}[hbt]
    \centering
    \subfloat{
        \includegraphics[width=0.4\linewidth,frame]{imagens/beamer0}}
    \subfloat{
    \includegraphics[width=0.4\linewidth,frame]{imagens/beamer1}}
\\
    \subfloat{
    \includegraphics[width=0.4\linewidth,frame]{imagens/beamer2}}
    \subfloat{
    \includegraphics[width=0.4\linewidth,frame]{imagens/beamer3}}

    \caption{Quatro passos de uma animação criada pelo \texttt{beamer}. Ele gera um slide em PDF para cada passo, porém mantém apenas um item na agenda do cabeçalho e nos \textit{bookmarks} do arquivo \texttt{.pdf}.}
    \label{fig:beamer}
\end{figure}

\begin{lstlisting}[language=TeX,caption={Código \LaTeX\ para gerar os slides da Figura \ref{fig:beamer}},label={list:beamer}]
\subsection{A interface do KNIME}
\begin{frame}{A interface do KNIME}
    %   \TPGrid[40mm,20mm]{10}{5}
    \only<1->{
        \centering
        \includegraphics[width=\linewidth]{Images/Interface1}
    }
    \setlength{\TPHorizModule}{\textwidth}
    \setlength{\TPVertModule}{\textheight}
    \only<2>{
        \begin{textblock*}{20mm}(60mm,40mm)\color{white}
            Área do Workflow
    \end{textblock*}}
    \only<3>{
        \begin{textblock*}{20mm}(20mm,40mm)
            \color{white} Explorador
    \end{textblock*}}
    \only<4>{
        \begin{textblock*}{30mm}(20mm,70mm)
            \color{white}Nós disponíveis
    \end{textblock*}}
\end{frame}
\end{lstlisting}
\section{Slides de Aulas}

Nesta seção tratamos exclusivamente de slides destinados a apresentação de aulas completas sobre um assunto específico. Esse tipo de sequência é comum para professores.

\subsection{Que slides ter}

Os seguintes slides são recomendados para uma boa aula:
\begin{outline}
    \1 \textbf{Título da aula}, como na Figura \ref{fig:titulo};
    \1 \textbf{Objetivo da aula};
    \1 \textbf{Revisão} do que é necessário para entender a aula;
    \2 \textbf{Contextualizando} a aula curso
    \1 \textbf{Habilidades} específicas que serão aprendidas;
    \1 \textbf{Agenda} (ou Sumário);
    \2 A agenda ou sumário divide a aula em seções;
    \1 Um slide de \textbf{título para cada seção};
    \2 Pode ser o slide da agenda colocando ênfase na seção atual, como na Figura \ref{fig:meio};
    \1 Slides de conteúdo;
    \2 Não esqueça de uma motivação quando necessário;
    \2 Não esqueça do contexto histórico do que está sendo ensinado;
    \2 Não esqueça de definições
    \1 Pelo menos um slide com um exercício
    \2 Passar uma atividade de aprendizagem pós aula também é interessante, mesmo que ela nunca seja feita;
    \1 Slide de resumo, \textbf{``o que vimos hoje''}
    \2 Esse slide, ou slides, devem fechar a aula. Se necessário, por estar sobrando tempo, indique que agora, para reforçar, serão feitos ou discutidos exercícios, e siga por esse caminho até o fim do tempo;
    \1 \textbf{Referências} bibliográficas;
    \1 \textit{Preview} da \textbf{próxima aula}
\end{outline}



\begin{figure}[ht]
    \centering
    \includegraphics[width=\tam\linewidth]{imagens/capa.png}
    \caption{Um slide de título.}
    \label{fig:titulo}
\end{figure}

Também é possível apresentar um slide de exercício no meio da aula. Nesse momento você deve ``quebrar a quarta parede'' e falar algo como ``normalmente eu daria 5 minutos e corrigiria o exercício, como vou fazer agora''.

Outra boa sugestão é ter um slide, no início, que leve a pensar sobre o conteúdo da aula. Esse slide pode mostrar um problema real onde a técnica poderia ser aplicada, sendo algo do tipo ``como vocês fariam para fazer x?''. Isso seria adequado para uma aula onde se ensina o método PERT/CPM para calcular prazos de um projeto. Já em uma aula de programação inicial, que vai usar exemplos numéricos, poderia ser proposto um problema numérico, como achar números primos.

Ao mostrar um problema é interessante mostrar como ele pode se complicar. Ao mostrar um método de fazer algo que suplantou outro anterior, é interessante mostrar os problemas que o anterior tinha. Deve haver cuidado, porém, na estimativa de tempo, o contexto histórico deve ser limitado a motivação. Se começar do início de tudo, você acabará tendo menos tempo para falar do assunto que deve abordar e poderá perder pontos.

\begin{figure}[hb]
    \centering
    \includegraphics[width=\tam\linewidth]{imagens/agendadomeio.png}
    \caption{Um slide mostrando a parte que será falada da agenda, com as outras partes acinzentadas. Criada usando o \LaTeX\  e o \texttt{beamer}.}
    \label{fig:meio}
\end{figure}

Eu agora também crio mais um slide, que fala sobre a metodologia da aula, e o tamanho da aula em slides e em tempo, como na Figura \ref{fig:metodologia}. Esse slide também mostra como símbolos podem ser usados para passar mensagens. Julgo ser  uma boa ideia mostrar isso também, inclusive porque não é uma prática comum entre os professores e pode surpreender positivamente.

\begin{figure}[htb]
    \centering
    \includegraphics[width=\tam\linewidth]{imagens/metodologia.png}
    \caption{Slide informando o aluno como vai ser a aula.}
    \label{fig:metodologia}
\end{figure}




\subsection{Estilo dos slides de aula}

A melhor estratégia para o estilo dos slides de aula são o fundo branco, letras escuras, e cores para ressaltar. Isso se adequa bem tanto a salas bem iluminadas quanto a salas escuras, para todo tipo de projetor. A Figura \ref{fig:teximag}, apesar de usar o forte grená, me parece bastante adequada. As outras figuras  mostram outros modelos que eu uso e sinto adequados para uma aula. As cores azuis e cinzas, porém, são mais ``fracas'' e podem levar a um pouco de monotonia.

\begin{figure}[hbt]
    \centering
    \includegraphics[width=\tam\linewidth]{imagens/slidecomimage.png}
    \caption{Slide com texto e imagem}
    \label{fig:teximag}
\end{figure}


\section{Slides para defesas}

Slides de defesa serão usados em uma ocasião muito formal, e devem seguir as recomendações gerais e ainda as recomendações de estilo para slides de aula, só que com mais cuidado para não causar estranheza à banca.

\subsection{Que slides ter}

A apresentação da tese deve ter como foco a apresentação do trabalho. Para os 50 minutos usados na COPPE, recomendo que pelo menos 50\% do tempo seja usado com o que o aluno fez. A conclusão pode ser rápida, mas não está incluída nesse tempo. O candidato deve tomar cuidado para que o tratametno da revisão e dos trabalhos correlatos não assuma a predominância da aprensetação.

Recomenda-se que os seguintes slides sejam usados:
\begin{outline}
    \1 \textbf{Título da dissertação ou tese}, como na Figura \ref{fig:titulo}, contendo ainda o nome do orientado e do orientador;
    \1 \textbf{Agenda} (ou Sumário);
    \1 Um slide de \textbf{título para cada seção};
    \2 Pode ser o slide da agenda colocando ênfase na seção atual, como na Figura \ref{fig:meio};
    \2 Slides de conteúdo, incluindo obrigatoriamente;
    \2 As seguintes seções/slides são obrigatórios
        \3 \textbf{Objetivo da tese e sub-objetivos};
        \3 \textbf{Motivação};
        \3 \textbf{Trabalhos correlatos}
        \3 \textbf{Conclusão};
        \3 \textbf{Trabalhos futuros};
    \1 \textbf{Referências} bibliográficas;
    \1 Slide de obrigado e abrindo para perguntas
\end{outline}

Na lista acima não tratamos dos slides principais, pois isto depende do estilo da tese. Em teses típicas do PESC são gerados artefatos computacionais que são avaliados de alguma forma. Nesse caso, costuma-se criar duas seções diferenciadas: a proposta e a avaliação. Eu costumo dividir ainda mais, usando três seções típicas, em proposta teórica, implementação, experimento e avaliação. Experimentos muito complicados podem exigir ainda a separação da explicação do experimento e do resultado dos mesmos, havendo então 4 partes.


\section{Slides de Apresentações de Artigos}

As apresentações de artigos são o momento onde podemos ser mais criativos com os slides. Isso vem da necessidade de chamar mais atenção em um menor espaço de tempo. É comum que uma apresentação dure apenas 15 minutos.


Uma boa estrutura de slides é:
\begin{itemize}
    \item Título, autores e indicação de como encontrar o artigo;
    \item Apresentação do grupo de pesquisa, já com indicação de contato;
    \item Agenda, que deve ser tratado bem rapidamente e está aqui apenas por uma questão formal;
    \item Apresentação do tema do artigo, do problema, dando motivação e justificativa;
    \item Revisão mínima dos trabalhos correlatos ou antecessores;
    \item Detalhamento da solução que o artigo traz;
    \item Contribuições;
    \item Trabalhos Futuros;
    \item Bibliografia, não será lida, apenas por uma questão formal;
    \item Agradecimentos, em especial os obrigatórios, como a CAPES e ao CNPq, e
    \item Slide de contato que ficará sendo apresentado enquanto se responde as perguntas e que pode indicar outros artigos dos autores relacionados ao tema.
\end{itemize}


\section{Ferramentas}

Esse texto considera 4 ferramentas possíveis para fazer slides acadêmicos:
\begin{itemize}
    \item Power Point
    \item Google Slides
    \item \LaTeX\  com \texttt{beamer}
\end{itemize}

\subsection{Power Point}

Se for usar o \textit{Power Point}, use um estilo e o siga. Evite criar caixas soltas de texto, já que os programa fornece estilos próprios.





\section*{Licença}


Este texto é distribuído com uma licença Creative Commons - Atribuição - NãoComercial - Compartilha Igual 4.0 Internacional.




\begin{center}
   \ccbyncsa
\end{center}

Você tem o direito de:
\begin{itemize}
    \item \textbf{Compartilhar} -- copiar e distribuir o material em qualquer suporte ou formato.
    \item \textbf{Adaptar} -- remixar, transformar, e criar a partir do material.
\end{itemize}

De acordo com os termos seguintes:
\begin{itemize}
    \item \textbf{Atribuição} -- Você deve dar o crédito apropriado, prover um link para a licença e indicar se mudanças foram feitas. Você deve fazê-lo em qualquer circunstância razoável, mas de nenhuma maneira que sugira que o licenciante apoia você ou o seu uso.
    \item \textbf{NãoComercial} --Você não pode usar o material para fins comerciais.
    \item \textbf{CompartilhaIgual} -- Se você remixar, transformar, ou criar a partir do material, tem de distribuir as suas contribuições sob a mesma licença que o original.
    \item \textbf{Sem restrições adicionais} -- Você não pode aplicar termos jurídicos ou medidas de caráter tecnológico que restrinjam legalmente outros de fazerem algo que a licença permita.
\end{itemize}

Mais informações podem ser encontradas em \url{https://creativecommons.org/licenses/by-nc-sa/4.0/deed.pt_BR}


\end{document}