\section{Uso de Seções}

A apresentação deve ser dividida em seções. As seções são apresentadas na agenda, e, de preferência, devem variar entre 5 e 7, não sendo menos que 3 e não ultrapassando 9.

Na maioria das apresentações, duas seções são essenciais: introdução e conclusão. Algumas partes da apresentação, mesmo que sendo obrigatória sua presença, não precisam ter seções indicadas, como a Bibliografia e, quando existir, a apresentação do autor, autores ou grupo de pesquisa. A agenda não é um sumário do documento apresentação, mas sim uma divisão da apresentação em etapas.

Cada seção deve possuir uma slide inicial, na forma de título ou de uma versão da agenda onde a versão atual está indicada de alguma forma. Um título de seção pode ter vários formatos, a Figura \ref{fig:sectitles} mostra um simples e três variações que experimentei em aulas diferentes. Em especial, a agenda do slide da Figura \ref{fig:stfancy} usa uma \textit{SmartArt} do \textit{Power Point} e permite usar imagens e funciona bem até 5 seções. Tanto esse formato, quanto do da Figura \ref{fig:st3} mostram toda a agenda de alguma forma.

\begin{figure}[htb]
    \centering
    \subfloat[]{
        \includegraphics[width=0.4\linewidth,frame]{imagens/sec1}
    }
    \subfloat[]{
        \includegraphics[width=0.4\linewidth,frame]{imagens/valor}
    }\\
    \subfloat[]{
        \includegraphics[width=0.4\linewidth,frame]{imagens/casosdeuso}\label{fig:st3}
    }
    \subfloat[]{
        \includegraphics[width=0.4\linewidth,frame]{imagens/sec2}
        \label{fig:stfancy}}
        \caption{Várias formas de fazer um título de seção}.
        \label{fig:sectitles}
    \end{figure}


 A Figura \ref{fig:meio} mostra um slide com uma versão da agenda levemente modificada por meio do uso de sombra nos itens já tratados, que é criada automaticamente no \texttt{beamer} com os comandos que aparecem na Listagem \ref{lst:autosec}.

\begin{lstlisting}[language=TeX,caption={Comando para títulos de seção automáticos no \texttt{beamer} com o tema Luebeck.},label={lst:autosec}]
    \AtBeginSection[]
    {\begin{frame}
            \frametitle{Onde Estamos?}
            \tableofcontents[currentsection,hideallsubsections ]
    \end{frame}}
\end{lstlisting}

\begin{figure}[hbt]
    \centering
    \includegraphics[width=\tam\linewidth,frame]{imagens/agendadomeio.png}
    \caption{Um slide mostrando a parte que será falada da agenda, com as outras partes acinzentadas. Criada usando o \LaTeX\  e o \texttt{beamer}.}
    \label{fig:meio}
\end{figure}




