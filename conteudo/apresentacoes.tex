\section{Slides de Apresentações de Artigos}

As apresentações de artigos são o momento onde podemos ser mais criativos com os slides. Isso vem da necessidade de chamar mais atenção em um menor espaço de tempo. É comum que uma apresentação dure apenas 15 minutos.


Uma boa estrutura de slides é:
\begin{itemize}
    \item Título, autores e indicação de como encontrar o artigo;
    \item Apresentação do grupo de pesquisa, já com indicação de contato;
    \item Agenda, que deve ser tratado bem rapidamente e está aqui apenas por uma questão formal;
    \item Apresentação do tema do artigo, do problema, dando motivação e justificativa;
    \item Revisão mínima dos trabalhos correlatos ou antecessores;
    \item Detalhamento da solução que o artigo traz;
    \item Contribuições;
    \item Trabalhos Futuros;
    \item Bibliografia, não será lida, apenas por uma questão formal;
    \item Agradecimentos, em especial os obrigatórios, como a CAPES e ao CNPq, e
    \item Slide de contato que ficará sendo apresentado enquanto se responde as perguntas e que pode indicar outros artigos dos autores relacionados ao tema.
\end{itemize}

