\section{Introdução}

Este documento é um guia com dicas para o uso de slides em apresentações no Programa de Engenharia de Sistemas e Computação.

Ele é construído como um guia básico, do mínimo necessário, e razoavelmente conservador.

Este material está disponível no GitHub, junto com material de apoio, como slides Power Point que eu uso e distribuo: \url{https://github.com/xexeo/DicasSlidesAcademicos}.

Outra questão importante neste documento: ele é criado por um usuário do \textit{Power Point} que tem experiência com o \texttt{beamer}, o estilo de apresentações do \LaTeX. Muitos softwares de apresentação são altamente compatíveis com o \textit{Power Point}, seguindo a mesma filosofia de trabalho, baseado no WYSIWYG. Outros softwares, como o Prezi\footnote{\url{https://prezi.com/}} podem exigir outra forma de pensar, porém muitas dicas dadas aqui continuam válidas. A Seção \ref{sec:ferramentas} discute um pouco os principais softwares e características importantes do seu uso.

Usarei o termo geral \textit{apresentação} para uma aula, defesa de tese ou apresentação de artigo em congresso.

Os slides são mostrados com frames para delimitar o seu tamanho.


