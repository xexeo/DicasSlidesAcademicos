\section{Slides para defesas}

Slides de defesa serão usados em uma ocasião muito formal, e devem seguir as recomendações gerais e ainda as recomendações de estilo para slides de aula, só que com mais cuidado para não causar estranheza à banca.

\subsection{Que slides ter}

A apresentação da tese deve ter como foco a apresentação do trabalho. Para os 50 minutos usados na COPPE, recomendo que pelo menos 50\% do tempo seja usado com o que o aluno fez. A conclusão pode ser rápida, mas não está incluída nesse tempo. O candidato deve tomar cuidado para que o tratametno da revisão e dos trabalhos correlatos não assuma a predominância da aprensetação.

Recomenda-se que os seguintes slides sejam usados:
\begin{outline}
    \1 \textbf{Título da dissertação ou tese}, como na Figura \ref{fig:titulo}, contendo ainda o nome do orientado e do orientador;
    \1 \textbf{Agenda} (ou Sumário);
    \1 Um slide de \textbf{título para cada seção};
    \2 Pode ser o slide da agenda colocando ênfase na seção atual, como na Figura \ref{fig:meio};
    \2 Slides de conteúdo, incluindo obrigatoriamente;
    \2 As seguintes seções/slides são obrigatórios
        \3 \textbf{Objetivo da tese e sub-objetivos};
        \3 \textbf{Motivação};
        \3 \textbf{Trabalhos correlatos}
        \3 \textbf{Conclusão};
        \3 \textbf{Trabalhos futuros};
    \1 \textbf{Referências} bibliográficas;
    \1 Slide de obrigado e abrindo para perguntas
\end{outline}

Na lista acima não tratamos dos slides principais, pois isto depende do estilo da tese. Em teses típicas do PESC são gerados artefatos computacionais que são avaliados de alguma forma. Nesse caso, costuma-se criar duas seções diferenciadas: a proposta e a avaliação. Eu costumo dividir ainda mais, usando três seções típicas, em proposta teórica, implementação, experimento e avaliação. Experimentos muito complicados podem exigir ainda a separação da explicação do experimento e do resultado dos mesmos, havendo então 4 partes.

